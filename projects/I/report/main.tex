% This is samplepaper.tex, a sample chapter demonstrating the
% LLNCS macro package for Springer Computer Science proceedings;
% Version 2.20 of 2017/10/04
%
\documentclass[runningheads]{llncs}
%
\usepackage{graphicx}
\usepackage{amsmath}
\usepackage[colorlinks=true, linkcolor=blue]{hyperref}
\usepackage{tabularx}
\newcolumntype{C}{>{\centering\arraybackslash}X}

\begin{document}

\title{Association Rules and Prediction Rules for Financial Data Mining}
\author{Yifei Fan}
\authorrunning{Y. Fan}
\institute{HKUST, Hong Kong SAR, China\\
    \email{yfanba@ust.hk}}
\maketitle

\begin{abstract}
    In this project, we applied association rules and prediction rules to financial data mining.
    We first downloaded the data and preprocessed the data. 
    To smooth the data and reduce the noise, we computed the exponential moving average.
    To figure out the statistical features of the data, we computed the CDF, PDF, and some descriptive statistics.
    We also carried out the Mean-Variance analysis to find out the minimum-risk portfolio.
    Then we digitized the time series and split the data into the training set and test set.
    We found five-day association rules and chose the best 10 rules with the highest confidence using the training set.
    We compared them with the best 10 rules with the highest profit using the test set to verify the association rules.
    Finally, we also carried out further analysis and discussion on association rules with different kinds of mean and RPF.\\
    In conclusion, all of these prove that the designer of this project created a meanless and tedious project, 
    which is just a step-by-step big homework without learning something new.\\
    The code of this project is available at \url{https://algebra-fun.github.io/MSDM5058/projects/I/code/}.
    \keywords{Association Rules  \and Prediction Rules \and Financial Data Mining.}
\end{abstract}

\section{Data Preprocessing}
In this project, we downloaded the data of
daily closing price of the time series of "MSFT" and "GOOGL"
started from 2005-01-01 to 2023-03-01 from Yahoo Finance~\cite{yahoo_finance}
using yfinance~\cite{yfinance} package.
\subsection{Compute log return}
In order to make the data more suitable for financial data mining,
we first computed the log return of the closing price of each stock.

Let time start at $t = 0$. Denote the stock's time series by $S(t)$, then compute its daily log return $X(t)$ with

\begin{equation}
    X(t) = ln \left[\frac{S(t)}{S(t-1)}\right].
\end{equation}

The daily log return of two time series are shown in Fig.~\ref{fig:log_return}.

\begin{figure}
    \centering
    \includegraphics[width=0.8\textwidth]{../img/log_return.pdf}
    \caption{Daily log return of two stocks.}
    \label{fig:log_return}
\end{figure}

\section{Exponential moving average}
In order to smooth the data and reduce the noise, we computed the exponential moving average of the price of each stock.

Define the stock's $\omega$-day exponential moving average (EMA) at time $t$ as
$$
    M(t, \omega) = aS(t) + (1 - a)M(t - 1, \omega)
$$
for some smoothing constant $a$. The base case is $M(0, \omega) ≡ S(0)$.

\subsection{First case $a=1/\omega$}

In first case, we set $a = 1/\omega$ and compute the $\omega$-day EMA of the log return of each stock.
The plots are shown in Fig.~\ref{fig:ema1}.

\begin{figure}
    \centering
    \includegraphics[width=0.7\textwidth]{../img/ema1.pdf}
    \caption{Exponential moving average of two stocks in first case.}
    \label{fig:ema1}
\end{figure}

\subsection{Second case $a=2/(\omega + 1)$}

In second case, we set $a = 2/(\omega + 1)$ and compute the $\omega$-day EMA of the log return of each stock.
The plots are shown in Fig.~\ref{fig:ema2}.

\begin{figure}
    \centering
    \includegraphics[width=0.7\textwidth]{../img/ema2.pdf}
    \caption{Exponential moving average of two stocks in second case.}
    \label{fig:ema2}
\end{figure}

\subsection{Discussion on effect of parameter $\omega$ and $a$}

The parameter $a$ is the smoothing constant. The larger the value of $a$ is, the more weight is given to the most recent data point.
Then curve of the EMA will be more unsmoothed since the most recent data point contributes more noise to EMA.

In both first case and second, the $a$ is always in inverse proportion to $\omega$. Then when $\omega$ is larger, the value of $a$ is smaller so that the curve of the EMA will be more smoothed.

Meanwhile, we call the EMA as $\omega$-day EMA since the value is similar as the average of the $\omega$-day stock prices.
The larger the value of $\omega$ is, the more data points are used to compute the EMA. Then the curve of the EMA will be more smoothed.

\section{Cumulative Distribution Function}

In order to figure out the distribution of the log return of each stock, we computed the  cumulative distribution function (CDF) of the log return of each stock.

The ECDF of the log return of each stock is shown in Fig.~\ref{fig:ecdf}.

\begin{figure}
    \centering
    \includegraphics[width=0.8\textwidth]{../img/ecdf.pdf}
    \caption{ECDF of two stocks.}
    \label{fig:ecdf}
\end{figure}


From the plot, we can see that the distribution curve looks like a normal distribution curve.
Then we deside to use the logistic function to fit the curve.

\subsection{Logistic function}
We may fit $F_X (x)$ with a logistic function
\begin{equation}
    L(x) = \frac1{1 + \exp[-b(x-x^*)]}
\end{equation}
In physics, this is also called a Fermi-Dirac distribution.
\subsubsection*{What is $L(x^*)$ and hence your empirical $x^*$?}
$$
    L(x^*) = \frac12 .
$$

Thus, $x^*=L^{-1}(\frac12)=F_X^{-1}(\frac12)=quantile(X,\frac12)$.
\subsubsection*{What is $L^{'}(x^*)$ and hence your empirical $b$?}
$$
    L^{'}(x) = \frac{b\exp[-b(x-x^*)]}{(1+\exp[-b(x-x^*)])^2} .
$$
$$
    L^{'}(x^*) = \frac{b}{4} .
$$

Thus, $b=4L^{'}(x^*)=4F_X^{'}(x^*)\approx 4\frac{F_X(x^*+\delta x)-F_X(x^*-\delta x)}{2\delta x}$.

\subsubsection*{Plot the empirical $L(x)$ atop $F_X(x)$}

\hfill\\
The plot is shown in Fig.~\ref{fig:logistic}.

\begin{figure}
    \centering
    \includegraphics[width=0.8\textwidth]{../img/logistic_fit.pdf}
    \caption{Empirical $L(x)$ atop $F_X(x)$}
    \label{fig:logistic}
\end{figure}

From the plot, we can see that the logistic function fits the distribution curve well.

\subsection{Kolmogorov-Smirnov Test}

In order to check whether the logistic function fits the distribution curve well, we use the Kolmogorov-Smirnov test.

We can test if our fitting is good with the Kolmogorov-Smirnov test. Our null
hypothesis is that $L(x)$ fits $F_X (x)$ well. Define $D = \max_x֓|F_X(x) - L(x)|$.
If $\sqrt{N} D > \eta_{\alpha}$ , we reject the null hypothesis at a significance level $\alpha$. The
threshold $\eta_{\alpha}$ solves
$$
    \frac{\sqrt{2\eta_{\alpha}}}{\eta_{\alpha}}\sum_{k=1}^{\infty}\exp\left[-\frac{(2k-1)^2\pi^2}{8\eta_{\alpha}^2}\right]=1-\alpha.
$$

In this project, we just calculate the $\eta_{\alpha}$ with help of the function "kolmogi"~\cite{kolmogi} from "scipy" package.

\subsubsection*{Does the test reject the null hypothesis at $\alpha$ = 0.05?}

\hfill\\
The output of the KS-Test is shown below:

\begin{itemize}
    \item MSFT:\\
    Accept the null hypothesis at $\alpha=0.05 (\sqrt{N}D=0.1273,\eta_{\alpha}=0.5196)$.
    \item GOOGL:\\
    Accept the null hypothesis at $\alpha=0.05 (\sqrt{N}D=0.1375,\eta_{\alpha}=0.5196)$.
\end{itemize}

\noindent The KS-Test shows that the logistic function fits the distribution curve well, both for MSFT and GOOGL.

\section{Probability Density Function}

In order to figure out the probability density function of the log return of each stock, we computed the probability density function of the log return of each stock.

We can estimate X's PDF $F_X(x)$ with the derivative of its fitted
CDF. On the other hand, we can estimate $F_X(x)$ with a $k$-bin normalized his-
togram, where each bin is $h = (\max x - \min x)/k$ units wide. In general, the
ith bin measures the frequency of $x \in [\min x + (i - 1)h, \min x + ih)$.

\begin{itemize}
    \item Plot $L^{'}(x)$ for $L(x)$ fitted in last section.
    \item Plot three $k$-bin histograms for $k = 20, 100$, and $400$.
\end{itemize}

The plot is shown in Fig.~\ref{fig:epdf}.
\begin{figure}
    \centering
    \includegraphics[width=0.8\textwidth]{../img/epdf.pdf}
    \caption{Empirical PDF}
    \label{fig:epdf}
\end{figure}

\subsection{Discussion on the effect of $k$}

The larger the $k$ is, the more accurate the histogram is, since the histogram is more smooth and the bar is more close to the definition of density.
However, the larger the $k$ is, the more bins there are, which means the more computation is needed.

\section{Descriptive Statistics}

In order to find the feature of Statistics of the log return of each stock, we computed the mean, variance, sharp-ratio of the log return of each stock.

We compute the two return rates' all-time means $\{\mu_1, \mu_2\}$, variances $\{\sigma_1^2, \sigma_2^2\}$
Sharpe ratios $\{\gamma_1, \gamma_2\}$. These statistics are shown in Table.~\ref{tab:stats1}\\
And the covariance $\sigma_{12}=0.000195$.

\begin{table}
    \centering
    \caption{The mean, variance, sharp-ratio of each stock(all-time).}
    \label{tab:stats1}
    \begin{tabularx}{\textwidth}{|C|C|C|C|}
        \hline
              & Mean     & Variance & Sharpe Ratio \\
        \hline
        MSFT  & 0.000566 & 0.000302 & 0.032541     \\
        GOOGL & 0.000630 & 0.000359 & 0.033221     \\
        \hline
    \end{tabularx}
\end{table}

In order to see whether the time series is stationary along the time(i.e. the statistics don't rely on time variable), we computed these statistics only using the data on the $K$ most recent days

\begin{itemize}
    \item When $k=30$, the statistics of each stock are shown in Table.~\ref{tab:stats2}. And the covariance $\sigma_{12}=0.000409$.
    \item When $k=100$, the statistics of each stock are shown in Table.~\ref{tab:stats3}. And the covariance $\sigma_{12}=0.000514$.
    \item When $k=300$, the statistics of each stock are shown in Table.~\ref{tab:stats4}. And the covariance $\sigma_{12}=0.000438$.
\end{itemize}

\begin{table}
    \centering
    \caption{The mean, variance, sharp-ratio of each stock($k=30$).}
    \label{tab:stats2}
    \begin{tabularx}{\textwidth}{|C|C|C|C|}
        \hline
              & Mean      & Variance & Sharpe Ratio \\
        \hline
        MSFT  & 0.001474  & 0.00041  & 0.072799     \\
        GOOGL & -0.000754 & 0.00088  & -0.025415    \\
        \hline
    \end{tabularx}
\end{table}
\begin{table}
    \centering
    \caption{The mean, variance, sharp-ratio of each stock($k=100$).}
    \label{tab:stats3}
    \begin{tabularx}{\textwidth}{|C|C|C|C|}
        \hline
              & Mean      & Variance & Sharpe Ratio \\
        \hline
        MSFT  & 0.000075  & 0.000554 & 0.003179     \\
        GOOGL & -0.001210 & 0.000704 & -0.045574    \\
        \hline
    \end{tabularx}
\end{table}
\begin{table}
    \centering
    \caption{The mean, variance, sharp-ratio of each stock($k=300$).}
    \label{tab:stats4}
    \begin{tabularx}{\textwidth}{|C|C|C|C|}
        \hline
              & Mean      & Variance & Sharpe Ratio \\
        \hline
        MSFT  & -0.000842 & 0.000474 & -0.038648    \\
        GOOGL & -0.001575 & 0.000594 & -0.064624    \\
        \hline
    \end{tabularx}
\end{table}

\section{Mean-Variance Analysis}

We would like to perform a mean-variance analysis on $X_1(t)$ and $X_2(t)$ and
accordingly construct the minimum-risk portfolio $S_p(t) = pS_1(t)+(1-p)S_2(t)$  for some fraction of investment $p\in[0, 1]$.
We determine $p$ according to the stocks' all-time statistics by minimize the risk of the portfolio $S_p(t)$.
Then we get the optimal $p=0.60467981$ and the portfolio $S_p(t)=0.6047S_1(t)+0.3953S_2(t)$.

The plot of the portfolio $S_p(t)$ is shown in Fig.~\ref{fig:portfolio}.

\begin{figure}
    \centering
    \includegraphics[width=0.8\textwidth]{../img/min_risk.pdf}
    \caption{The resultant portfolio's $S_p(t)$ atop $S_1(t)$ and $S_2(t)$.}
    \label{fig:portfolio}
\end{figure}

From the plot, we can see that although the portfolio $S_p(t)$ is not as benefit as $S_1(t)$, it indeed reduces the risk.

\subsection{A K-Day Analysis}
As the relevance of old data should decay, it is more sensible to consider the
stocks' performance on the K most recent days only. In other words, we only
infer information from $\{X(t-\tau)|\tau\in(0,K-1)\}$ at every moment $t$.

Hence, the fraction of investment, now denoted by $p(t, K)$, varies with time
and depends on K. Complete the following tasks for $K = 30, 100, and 300$.

\begin{itemize}
    \item Compute $p(t, K)$.
    \item Compute the resultant portfolio $S_p(t,K)$ atop $S_1(t)$ and $S_2(t)$
    \item Compute $S_p(t,K)$'s K-day Sharpe ratio $\gamma_p(t,K)=\mu_p(t,K)/\sigma_p(t,K)$.
\end{itemize}

The plot is shown in Fig.~\ref{fig:portfolio_k}.

\begin{figure}
    \centering
    \includegraphics[width=0.8\textwidth]{../img/k_day_analysis.pdf}
    \includegraphics[width=0.8\textwidth]{../img/k_day_analysis_Sp.pdf}
    \caption{The K-Day Analysis}
    \label{fig:portfolio_k}
\end{figure}

\subsubsection*{Discussion on the effect of $K$ on the portfolio's performance}
\hfill\\
From the plot, we can see that the stock log return of the portfolio $S_p(t,K)$ is more stable when $K$ is larger.
However, the sharp-ratio $\gamma_p(t,K)$ is smaller when $K$ is larger.
Which means that when $K$ is larger, although risk is reduced, the benefit is also reduced, leading to the reduction on shart-ratio.

\section{Digitization of Time Series}
Now focus on the first stock, so its subscript “1” is hereafter dropped and
implied. Digitize X(t) as Y(t) with three alphabets, viz. D for “down”, U for
“up”, and H for “hold”.
\begin{equation}
    Y(t)=
    \begin{cases}
        D, & X(t)<-\epsilon, \\
        U, & X(t)>+\epsilon, \\
        H, & otherwise.
    \end{cases}
\end{equation}
Here let us use $\epsilon = 0.002$, whereas you may of course choose other values in
real applications.
\subsection*{Calculate the probability $P[Y(t)=y]$ for $y \in \{D, U, H\}$}

The result is shown in Table~\ref{tab:p}.

\begin{table}
    \centering
    \caption{The probability $P[Y(t)=y]$ for $y \in \{D, U, H\}$}
    \label{tab:p}
    \begin{tabularx}{\textwidth}{|C|C|}
        \hline
          & probability \\
        \hline
        U & 0.438827    \\
        D & 0.408186    \\
        H & 0.152988    \\
        \hline
    \end{tabularx}
\end{table}

\subsection*{Calculate the conditional probability $P[Y(t+1)=y_1|Y(t)=y_2]$ for all
    nine possible pairs of $(y_1,y_2) \in \{D, U, H\} \times \{D, U, H\}$}

The result is shown in Table~\ref{tab:p12}.

\begin{table}
    \centering
    \caption{The conditional probability $P[Y(t+1)=y_1|Y(t)=y_2]$.}
    \label{tab:p12}
    \begin{tabularx}{\textwidth}{|C|C|C|}
        \hline
        $y_1$ & $y_2$ & probability \\
        \hline
        D     & U     & 0.432918    \\
        U     & D     & 0.455764    \\
        U     & U     & 0.413965    \\
        D     & D     & 0.398928    \\
        U     & H     & 0.464950    \\
        H     & U     & 0.152618    \\
        H     & D     & 0.145308    \\
        D     & H     & 0.360515    \\
        H     & H     & 0.174535    \\
        \hline
    \end{tabularx}
\end{table}

\section{Association Rules}
We would like to find out a five-day pattern A that associates well with an
immediate down (D). Formally, a rule R can be written as
R : A = {Y(t - 4), Y(t - 3), Y(t - 2), Y(t - 1), Y(t)} → Y(t + 1) = D (6)
or just R : A → D. Since the consequence of R is always D, we may simplify
our notation by labelling R with only the five alphabets of A; for example, the
rule R = UUUUU predicts a down after five consecutive ups. There are 3Θ =
243 possible rules.
\begin{itemize}
    \item Divide Y(t) into the “past” and the “future” at $t = M \approx 3N /4$. We hope to
          get good association rules by examining the history on the previous 3000
          days and then apply them on the coming 1000 days to make a profit.
    \item Calculate the confidence of all the 243 rules in the past, then report the 10
          rules with the highest confidence. Denote them as $\{R_{conf}\}_p$.
\end{itemize}

The result is shown in Table~\ref{tab:conf}.

\begin{table}
    \centering
    \caption{The 10 most confident rules $\{R_{conf}\}_p$.}
    \label{tab:conf}
    \begin{tabularx}{\textwidth}{|C|C|C|C|C|C|}
        \hline
        $y_{t}$ & $y_{t-1}$ & $y_{t-2}$ & $y_{t-3}$ & $y_{t-4}$ & probability \\
        \hline
        U       & D         & D         & H         & H         & 1.000000    \\
        U       & D         & H         & H         & H         & 1.000000    \\
        D       & H         & H         & D         & H         & 1.000000    \\
        H       & U         & H         & D         & H         & 1.000000    \\
        H       & U         & H         & H         & H         & 1.000000    \\
        D       & D         & H         & H         & H         & 1.000000    \\
        H       & D         & H         & D         & U         & 0.833333    \\
        U       & D         & H         & U         & H         & 0.800000    \\
        D       & D         & H         & D         & H         & 0.800000    \\
        U       & H         & D         & D         & U         & 0.785714    \\
        \hline
    \end{tabularx}
\end{table}

\section{Verification of Association Rules}
We may verify a rule's goodness by doing a betting experiment with the future data like this: as time passes, we bet on an immediate down every day.
If a down indeed comes out, we earn $u$, and we ascribe the profit to the just appeared five-day pattern A and its rule R:A → D; otherwise, we lose $v$,
and we similarly ascribe the loss to the just appeared pattern.
Consider $u = 1$ and $v = 0$ for simplicity. In this regard, a rule's profit is merely the number of times that it works.
Record the profits due to all the 243 rules.
\subsection*{Find the 10 most profitable rules. Denote them as $\{R_{expt}\}_f$.}
The result is shown in Table~\ref{tab:expt}.
\begin{table}
    \centering
    \caption{The 10 most profitable rules $\{R_{expt}\}_f$.}
    \label{tab:expt}
    \begin{tabularx}{\textwidth}{|C|C|C|C|C|C|}
        \hline
        $y_{t}$ & $y_{t-1}$ & $y_{t-2}$ & $y_{t-3}$ & $y_{t-4}$ & profit \\
        \hline
        U       & U         & D         & D         & U         & 14     \\
        U       & D         & D         & U         & U         & 12     \\
        U       & D         & U         & U         & U         & 12     \\
        U       & U         & U         & U         & D         & 12     \\
        D       & U         & D         & U         & U         & 11     \\
        U       & U         & U         & U         & U         & 10     \\
        U       & U         & D         & U         & U         & 10     \\
        D       & U         & U         & D         & U         & 9      \\
        U       & D         & U         & D         & D         & 9      \\
        U       & U         & D         & U         & D         & 9      \\
        \hline
    \end{tabularx}
\end{table}

\subsection*{How many rules does $\{R_{conf}\}_p$ share with $\{R_{expt}\}_f$?}

Unfortunately, none of the 10 most confident rules $\{R_{conf}\}_p$ share with the 10 most profitable rules $\{R_{expt}\}_f$.

This result is not surprising. The confidence of a rule is merely the number of times that it works in the past. It does not guarantee that the rule will work in the future.
Meanwhile in this simplified setting $u=1$ and $v=0$, there is not punishment for a wrong prediction. Thus the most profitable rules may be the rule with low confidence but high support.
The reason those rules are profitable is that they have a high support, which means the higher support they have, the more time they can win and get more profit.

In conclusion, the poor setting $u=1$ and $v=0$ leads to this bad result. If we want to get a good result, we should set $u$ and $v$ to be more realistic values.

\subsection*{Compute the correlation between a rule's confidence in the past and its
    profit in the future.}

The result is only $0.003803$, a very low correlation.

\subsection*{Discussion on the discrepancies between $\{R_{conf}\}_p$ and $\{R_{expt}\}_f$. What are the
    potential flaws of assessing a rule with its confidence?
}

Actually, there is all discrepancies between $\{R_{conf}\}_p$ and $\{R_{expt}\}_f$. They don't share any common rule.

As we discussion above, the poor setting $u=1$ and $v=0$ leads to this bad result.

The flaws is quite obvious.

\begin{itemize}
    \item The confidence of a rule is merely the number of times that it works in the past. It does not guarantee that the rule will work in the future.
    \item Only higher confidence can't guarantee profitable in the future. We should consider the support of a rule. Only a rule with higher support and higher confidence can have a higher chance gain more profit in the future.
\end{itemize}

\section{Further Analysis of Association Rules}

\subsection{Geometric mean and Arithmetic Mean}
Let us first boldly assume that all rules obey $s_p= s_f$ and $c_p= c_f$ , yielding $\pi\sim s_pc_p$. Since the ranking of 𝜋 does not change when we take a square root on the right-hand side, we may predict a rule’s goodness with the geometric mean between its support and confidence. Out of curiosity, we may also guess whether their arithmetic mean is useful.
\subsubsection*{Report the 10 rules with the highest geometric mean between support and confidence in the past, i.e. $\sqrt{s_p c_p}$. Denote them as $\{R_{geo}\}_p$.}
\hfill\\
The result is shown in Table~\ref{tab:geo}.
\begin{table}
    \centering
    \caption{The 10 rules with the highest geometric mean between support and confidence in the past.}
    \label{tab:geo}
    \begin{tabularx}{\textwidth}{|C|C|C|C|C|C|}
        \hline
        $y_{t}$ & $y_{t-1}$ & $y_{t-2}$ & $y_{t-3}$ & $y_{t-4}$ & geometric mean \\
        \hline
        D       & U         & D         & D         & U         & 5.385165       \\
        D       & U         & U         & D         & U         & 5.291503       \\
        U       & U         & D         & U         & U         & 5.291503       \\
        D       & U         & D         & D         & D         & 5.196152       \\
        U       & D         & D         & U         & D         & 5.196152       \\
        U       & D         & D         & D         & U         & 5.099020       \\
        U       & D         & U         & D         & U         & 5.099020       \\
        D       & D         & U         & D         & D         & 5.000000       \\
        U       & D         & D         & U         & D         & 4.898979       \\
        U       & U         & D         & D         & D         & 4.690416       \\
        \hline
    \end{tabularx}
\end{table}

\subsubsection*{Report the 10 rules with the highest arithmetic mean between support and confidence in the past, i.e. $\frac{s_p+c_p}2$. Denote them as $\{R_{ari}\}_p$.}
\hfill\\
The result is shown in Table~\ref{tab:ari}.
\begin{table}
    \centering
    \caption{The 10 rules with the highest arithmetic mean between support and confidence in the past.}
    \label{tab:ari}
    \begin{tabularx}{\textwidth}{|C|C|C|C|C|C|}
        \hline
        $y_{t}$ & $y_{t-1}$ & $y_{t-2}$ & $y_{t-3}$ & $y_{t-4}$ & arithmetic mean \\
        \hline
        D       & U         & D         & D         & U         & 30.737705       \\
        U       & D         & U         & D         & U         & 30.713115       \\
        D       & U         & U         & D         & U         & 30.233333       \\
        D       & D         & U         & D         & D         & 29.711864       \\
        U       & D         & D         & D         & U         & 28.232143       \\
        U       & D         & D         & U         & D         & 27.745455       \\
        U       & U         & D         & U         & U         & 26.764151       \\
        U       & U         & D         & U         & D         & 26.698113       \\
        D       & U         & D         & U         & D         & 25.170000       \\
        U       & U         & D         & D         & D         & 24.724490       \\
        \hline
    \end{tabularx}
\end{table}

\subsubsection*{How many rules do $\{R_{geo}\}_p$ and $\{R_{ari}\}_p$ share with $\{R_{expt}\}_f$ ? Do the two means assess a rule better than confidence?}

\begin{equation}
    \label{eq:geo_ari}
    \begin{aligned}
        \{R_{ari}\}_p\cup \{R_{expt}\}_f & = \{(D, U, U, D, U),(U, U, D, U, D),(U, U, D, U, U)\} \\
        \{R_{geo}\}_p\cup \{R_{expt}\}_f & = \{(D, U, U, D, U), (U, U, D, U, U)\}
    \end{aligned}
\end{equation}

From \eqref{eq:geo_ari}, we can see that the number of shared rules between $\{R_{geo}\}_p$ and $\{R_{expt}\}_f$ is only 2, while the number of shared rules between $\{R_{ari}\}_p$ and $\{R_{expt}\}_f$ is 3.
Both of the number of shared rules is bigger than the confidence's which is 0.
Therefore, we can conclude that both the arithmetic mean and geometric mean assess a rule better than the confidence since they consider the effect of support.
However, we still can't conclude that the arithmetic mean assesses a rule better than the geometric mean since the geometric mean is more theoretical senseable than the arithmetic mean.

\subsection{A Generalized Mean}
Now we will account for the discrepancy between the past and the future, so $s_p\neq s_f$ and $c_p\neq c_f$. We are pessimistic, so we expect that a rule's confidence depreciates over time, but we are also optimistic, so we expect that a more confident rule in the past remains more confident in the future.
The two assumptions combine to suggest $c_f=c_p^m$ for some $m > 1$. With the scale of $\pi$ maintained, we may formulate $\pi$ with a generalized mean
$$
    \pi=(s_pc_p^m)^{\frac1{1+m}}=s_p^\lambda c_p^{1-\lambda}
$$
for some tuning parameter $\lambda \equiv  1/(1 + m) \in [0, 1]$.
As $\lambda$ rises, the emphasis of $s_p^\lambda c_p^{1-\lambda}$ smoothly slides from support to confidence.
When $\lambda$ strikes 1/3,$\pi=\sqrt[3]{s_pc_p^2}=\sqrt[3]{r_p}$, where $r_p$ is the rule's rule power factor (RPF).

\subsubsection*{Report the 10 rules with the highest RPF in the past. Denote them as $\{R_{RPF}\}_p$.}
\hfill\\
The result is shown in Table~\ref{tab:rpf}.
\begin{table}
    \centering
    \caption{The 10 rules with the highest RPF in the past.}
    \label{tab:rpf}
    \begin{tabularx}{\textwidth}{|C|C|C|C|C|C|}
        \hline
        $y_{t}$ & $y_{t-1}$ & $y_{t-2}$ & $y_{t-3}$ & $y_{t-4}$ & RPF      \\
        \hline
        D       & U         & D         & D         & D         & 2.476445 \\
        U       & U         & D         & U         & U         & 2.454785 \\
        D       & U         & D         & D         & U         & 2.397850 \\
        U       & D         & D         & U         & D         & 2.366582 \\
        D       & U         & U         & D         & U         & 2.355347 \\
        U       & D         & D         & U         & U         & 2.335360 \\
        U       & D         & D         & D         & U         & 2.293962 \\
        D       & D         & U         & U         & D         & 2.289428 \\
        U       & D         & U         & D         & U         & 2.229490 \\
        D       & D         & U         & D         & D         & 2.196221 \\
        \hline
    \end{tabularx}
\end{table}

\subsubsection*{How many rules does $\{R_{RPF}\}_p$ share with $\{R_{expt}\}_f$ ? Hence, does RPF assess a rule better than confidence?}

\begin{equation}
    \label{eq:rpf_expt}
    \begin{aligned}
        \{R_{RPF}\}_p\cup \{R_{expt}\}_f & = \{(D, U, U, D, U),(U, D, D, U, U),(U, U, D, U, U)\}
    \end{aligned}
\end{equation}

From \eqref{eq:rpf_expt}, we can see that the number of shared rules between $\{R_{RPF}\}_p$ and $\{R_{expt}\}_f$ is 3, which is bigger than the confidence's which is 0. 
Therefore, we can conclude that RPF assesses a rule better than the confidence since it considers the effect of support and meanwhile using $c_f=c_p^2$ to predict the confidence in the future.


\subsubsection*{Do you think there will be an optimal value for $\lambda$ that best predict a rule's goodness? If yes, what may it be? If no, why not?}

\hfill\\
In my opinion, there will be an optimal value for $\lambda$ that best predict a rule's goodness in this practical application for specified data but there is no evidence in theory.

We carry out the experiment with $\lambda=linspace(0,1,101)$ to find out the best $\lambda$ for this task.

Then the optimal $\lambda$ we found is $1/4$. When $\lambda=1/4$, the number of shared rules between $\{R_{\lambda=1/4}\}_p$ and $\{R_{expt}\}_f$ is 4, which is bigger than 3. 
Then we conclude that the optimal $\lambda$ is $1/4$, which is better than the RPF setting $\lambda=1/3$.

The shared rules are shown in \eqref{eq:lambda_expt}.

\begin{equation}
    \label{eq:lambda_expt}
    \begin{aligned}
        \{R_{\lambda=1/4}\}_p\cup \{R_{expt}\}_f & = \{(D,U,U,D,U),(U,D,D,U,U),\\
        &(U,U,D,U,U),(U,U,U,U,U)\}
    \end{aligned}
\end{equation}

%
% ---- Bibliography ----
%
% BibTeX users should specify bibliography style 'splncs04'.
% References will then be sorted and formatted in the correct style.
%
% \bibliographystyle{splncs04}
% \bibliography{mybibliography}
%
\begin{thebibliography}{8}
    \bibitem{yahoo_finance}
    Yahoo! Finance, \url{https://finance.yahoo.com/}, last accessed 2023/04/15.

    \bibitem{yfinance}
    yfinance, \url{https://pypi.org/project/yfinance/}, last accessed 2023/04/15.

    \bibitem{kolmogi}
    scipy.special.kolmogi, \url{https://docs.scipy.org/doc/scipy/reference/generated/scipy.special.kolmogi.html}, last accessed 2023/04/15.

\end{thebibliography}
\end{document}
