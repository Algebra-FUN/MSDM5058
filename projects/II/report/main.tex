\documentclass[runningheads]{llncs}
\usepackage[T1]{fontenc}
\usepackage{graphicx}
\usepackage{color}
\usepackage{subcaption}
\usepackage{listings}
\usepackage[lighttt]{lmodern}
\usepackage{url}
\usepackage[colorlinks=true,urlcolor=blue]{hyperref}
\usepackage{booktabs}
\usepackage{multirow}
\usepackage{amsmath}
% Define a custom color
\definecolor{backcolour}{rgb}{0.95,0.95,0.92}
\definecolor{codegreen}{rgb}{0,0.6,0}

% \renewcommand\UrlFont{\color{blue}\rmfamily}
\begin{document}
\title{MSDM5058 Information Science\\
    Computational Project II:\\
    Portfolio Management Using Prediction Rules\\
    and Communication in Social Networks}
\titlerunning{MSDM5058 Project II}
\author{Yifei Fan}
\authorrunning{Y. Fan}
\institute{HKUST, Hong Kong SAR, China\\
    \email{\href{yfanba@ust.hk}{yfanba@ust.hk}}}
\maketitle

\begin{abstract}
    In this project, we applied prediction rules and communication in social networks carry out portfolio management and financial data mining.
    We first downloaded the data and preprocessed the data.
    The code of this project is available at \url{https://algebra-fun.github.io/MSDM5058/projects/II/}.
    \keywords{Prediction Rules \and Communication in Social Networks \and Portfolio Management.}
\end{abstract}


\section{Data Preprocessing}\label{sec:1}

In this project, we downloaded the data of
daily closing price of the time series of "GOOGL"~\cite{goolge}
started from 2007-01-01 to 2023-04-10 from Yahoo Finance
using yfinance~\cite{yfinance} package.
\subsection{Compute log return}
In order to make the data more suitable for financial data mining,
we first computed the log return of the closing price of each stock.

Let time start at $t = 0$. Denote the stock's time series by $S(t)$, then compute its daily log return $X(t)$ with

\begin{equation}
    X(t)=\ln \left[\frac{S(t)}{S(t-1)}\right].
\end{equation}

\begin{figure}[!htbp]
    \begin{center}
        \includegraphics[width=0.8\textwidth]{../code/img/q1_SX.pdf}
    \end{center}
    \caption{}
    \label{fig:q1sx}
\end{figure}

\begin{figure}[!htbp]
    \begin{subfigure}[t]{0.49\textwidth}
        \centering
        %		 include first image
        \includegraphics[width=\linewidth]{../code/img/q1_acf.pdf}
        \caption{Auto-correlation function with 50 days' lag.}
    \end{subfigure}
    \begin{subfigure}[t]{0.49\textwidth}
        \centering
        %		 include second image
        \includegraphics[width=\linewidth]{../code/img/q1_pacf.pdf}
        \caption{Partial auto-correlation function with 50 days' lag.}
    \end{subfigure}
    \caption{Auto-correlation function and partial auto-correlation function of the daily return.}
    \label{fig:q1acf}
\end{figure}


\begin{align}
    Y(t)=\left\{
    \begin{array}{ll}
        \text { D, } & \text { if } X(t)<-\varepsilon \\
        \text { U, } & \text { if } X(t)>+\varepsilon \\
        \text { H, } & \text { else }
    \end{array}
    \right.
\end{align}
where $\varepsilon=0.03$.


\section{Cumulative Distribution Function}\label{sec:2}
In this section, we plot the conditional CDF of $X$ given that $Y$ will become U and D one day later, which are denoted as
\begin{align}
    F_U(x) & = \text{CDF}[X(t)=x|Y(t+1)=U], \\
    F_D(x) & = \text{CDF}[X(t)=x|Y(t+1)=D].
\end{align}



\begin{figure}[!htb]
    \begin{center}
        \includegraphics[width=0.6\textwidth]{../code/img/q2.pdf}
    \end{center}
    \caption{ Cumulative distribution function of $X$ given that $Y$ will become U and D one day later.}
    \label{fig:q2}
\end{figure}


\section{Probability Density Function}\label{sec:3}


\begin{figure}[!htb]
    \begin{center}
        \includegraphics[width=0.6\textwidth]{../code/img/q3_dist.pdf}
    \end{center}
    \caption{Density plot of conditional PDFs}
    \label{fig:q3}
\end{figure}

The logistic function is defined as
\begin{equation}
    L(x)=\frac{1}{1+e^{-b(x-x^*)}},
\end{equation}
where $x^*$ is the inflection point and $b$ is the slope. The gaussian function is defined as
\begin{equation}
    G(x)=\frac{1}{\sigma \sqrt{2 \pi}} \exp \left[-\frac{(x-\mu)^{2}}{2 \sigma^{2}}\right],
\end{equation}
where $\mu$ is the mean and $\sigma$ is the standard deviation.




\section{Bayes Detector}\label{sec:4}


\begin{figure}[!htbp]
    \begin{subfigure}[t]{0.49\textwidth}
        \centering
        %		 include first image
        \includegraphics[width=\linewidth]{../code/img/q4_ld.pdf}
        \caption{Logistic function}
    \end{subfigure}
    \begin{subfigure}[t]{0.49\textwidth}
        \centering
        %		 include second image
        \includegraphics[width=\linewidth]{../code/img/q4_gau.pdf}
        \caption{Gaussian function}
    \end{subfigure}
    \caption{Bayes detector with different conditional PDFs. The red vertical line is the threshold.}
    \label{fig:q4}
\end{figure}


\section{Association Rules}\label{sec:5}




The result is shown in Table~\ref{tab:5}.
\begin{table}[!htbp]
    \centering
    \caption{Best Association Rules}
    \label{tab:5}
    \begin{tabular}{|c|c|c|}
        \hline
        \textbf{Days} & \textbf{Upward Rule} & \textbf{Downward Rule} \\ \hline
        1             & U                    & D                      \\ \hline
        5             & HUDHH                & HHHDD                  \\ \hline
    \end{tabular}
\end{table}


\section{Portfolio Management}\label{sec:6}

\begin{equation}
    V(t)=M(t)+N(t)S(t).
\end{equation}

If Bayes detector predicts $Y(t+1)$ is U and the antecedent of the best $k$-day upward rule $R_U$ occurs on day $t$, we buy shares to get
\begin{equation}\label{eq:61}
    \left\{\begin{array}{l}
        M(t) \quad \leftarrow M(t)-m \\
        N(t) \quad \leftarrow N(t)+m / S(t) \quad \text { for } \quad m=\gamma M(t) .
    \end{array}\right.
\end{equation}
Else if Bayes detector predicts $Y(t+1)$ is D and the antecedent of the best $k$-day downward rule $R_D$ occurs on day $t$, we sell shares to get
\begin{equation}\label{eq:62}
    \left\{\begin{array}{l}
        M(t) \quad \leftarrow M(t)+n S(t) \\
        N(t) \quad \leftarrow N(t)-n  \quad \text { for } \quad n=\gamma N(t) .
    \end{array}\right.
\end{equation}
Else we do nothing.



\begin{figure}[!htbp]
    \begin{subfigure}[t]{0.49\textwidth}
        \centering
        %		 include first image
        \includegraphics[width=\linewidth]{../code/img/q6_k_1.pdf}
        \caption{Use best 1-day rule and Bayes detector.}
    \end{subfigure}
    \begin{subfigure}[t]{0.49\textwidth}
        \centering
        %		 include second image
        \includegraphics[width=\linewidth]{../code/img/q6_k_5.pdf}
        \caption{Use best 5-day rule and Bayes detector.}
    \end{subfigure}
    \caption{}
    \label{fig:q6}
\end{figure}


\section{Transaction Cost}\label{sec:7}

Change Formula \ref{eq:61} and Formula \ref{eq:62} to the following equations to include transaction cost.
\begin{equation}
    \left\{\begin{array}{l}
        M(t) \quad \leftarrow M(t)-m \\
        N(t) \quad \leftarrow N(t)+(1-\xi)m / S(t) \quad \text { for } \quad m=\gamma M(t) .
    \end{array}\right.
\end{equation}
\begin{equation}
    \left\{\begin{array}{l}
        M(t) \quad \leftarrow M(t)+(1-\xi)n S(t) \\
        N(t) \quad \leftarrow N(t)-n  \quad \text { for } \quad n=\gamma N(t) .
    \end{array}\right.
\end{equation}
where $\xi$ is the tax.



\begin{figure}[!htbp]
    \begin{subfigure}[t]{0.49\textwidth}
        \centering
        %		 include first image
        \includegraphics[width=\linewidth]{../code/img/q7_(xi=0).pdf}
        % \caption{$\xi=0$, no tax.}
    \end{subfigure}
    \begin{subfigure}[t]{0.49\textwidth}
        \centering
        %		 include second image
        \includegraphics[width=\linewidth]{../code/img/q7_(xi=0.001).pdf}
        % \caption{$\xi=0.001$}
    \end{subfigure}
    ~
    \begin{subfigure}[t]{0.49\textwidth}
        \centering
        %		 include third image
        \includegraphics[width=\linewidth]{../code/img/q7_(xi=0.002).pdf}
        % \caption{}
    \end{subfigure}
    \begin{subfigure}[t]{0.49\textwidth}
        \centering
        %		 include third image
        \includegraphics[width=\linewidth]{../code/img/q7_(xi=0.005).pdf}
        % \caption{}
    \end{subfigure}
    \caption{Portfolio's value with different transaction cost. The blue and orange curve used $k=1$ and $k=5$ respectively. The trade time of each cases is $k=1$: 493 and $k=5$: 20.}
    \label{fig:q7}
\end{figure}


\section{Risk-Free Interest}\label{sec:8}

\begin{equation}
    M(t) \leftarrow M(t) \times (1+r)
\end{equation}
where $r$ is the risk-free interest rate.

To compare the portfolio��s performance with benchmark value, we define the ratio
\begin{equation}
    \rho(t; r) \equiv \frac{V(t; r)}{V_b(t; r)}
\end{equation}
where $V_b(t; r)=M(0) \times(1+r)^t$ is the value of the portfolio if we do not trade at all.



\begin{figure}[!htbp]
    \begin{subfigure}[t]{0.49\textwidth}
        \centering
        %		 include first image
        \includegraphics[width=\linewidth]{../code/img/q8_(r=0).pdf}
        % \caption{}
        % \label{fig:}
    \end{subfigure}
    \begin{subfigure}[t]{0.49\textwidth}
        \centering
        %		 include second image
        \includegraphics[width=\linewidth]{../code/img/q8_(r=0.0001).pdf}
        % \caption{}
        % \label{fig:}
    \end{subfigure}
    ~
    \begin{subfigure}[t]{0.49\textwidth}
        \centering
        %		 include first image
        \includegraphics[width=\linewidth]{../code/img/q8_(r=1e-05).pdf}
        % \caption{}
    \end{subfigure}
    \begin{subfigure}[t]{0.49\textwidth}
        \centering
        %		 include second image
        \includegraphics[width=\linewidth]{../code/img/q8_(r=5e-05).pdf}
        % \caption{}
    \end{subfigure}
    \caption{}
    \label{fig:q8}
\end{figure}


\section{Greed}\label{sec:9}



\begin{figure}[!htbp]
    \begin{subfigure}[t]{0.49\textwidth}
        \centering
        %		 include first image
        \includegraphics[width=\linewidth]{../code/img/q9_1.pdf}
        \caption{Portfolio's value with different $\gamma$.}
    \end{subfigure}
    \begin{subfigure}[t]{0.49\textwidth}
        \centering
        %		 include second image
        \includegraphics[width=\linewidth]{../code/img/q9_2.pdf}
        \caption{The effect of $\gamma$ to the final, highest and average value of portfolio.}
    \end{subfigure}
    \caption{ The effect of $\gamma$ to the portfolio's value across time and the final, highest and average value of portfolio.}
    \label{fig:q9}
\end{figure}


\section{Efficient Frontier}\label{sec:10}
% Right before transaction on day $t$, the portfolio's return amounts to $u=$ $A u_1+(1-A) u_2$, where $A=M(t) / V(t)$ is the proportion of money. If the signal of buying is observed, an investor spends $\widetilde{m}=\gamma_{\mathrm{U}} M(t)$ buying $(1-\xi) \widetilde{m} /$ $S(t)$ shares and changes $u$ to $u_{\mathrm{U}}=A_{\mathrm{U}} u_1+\left(1-A_{\mathrm{U}}\right) u_2$, tax rate is $\xi$. In contrast, if the signal of selling is observed, an investor sells $\tilde{n}=\gamma_{\mathrm{D}} N(t)$ shares to earn $(1-\xi) \tilde{n} S(t)$ and changes $u$ to $u_{\mathrm{D}}=A_{\mathrm{D}} u_1+\left(1-A_{\mathrm{D}}\right) u_2$. 
% $V(t)=M(t)+N(t)S(t)$

% Let $u$, $u_1$ and $u_2$ be the expected returns of the portfolio, $M(t)$ and $S(t)$. Before the transaction on day $t$, the portfolio's return amounts to $u = Au_1 + (1-A)u_2$, where $A=M(t)/V(t)$ is the proportion of money. 

% If the signal of buying is observed, an investor spends $\tilde{m}=\gamma_U M(t)$ buying $(1-\xi)\tilde{m}/S(t)$ shares making the portfolio $V(t)$ decrease $\xi\gamma_U M(t)$ and changes $u$ to $u_U=A_U u_1 + (1-A_U)u_2$,

% - Express $A_{\mathrm{U}}$ in terms of $\left\{\gamma_{\mathrm{U}}, V(t), M(t), \xi\right\}$ and $1-A_{\mathrm{D}}$ in terms of $\left\{\gamma_{\mathrm{D}}, V(t), N(t), S(t), \xi\right\}$
% - Express $A_{\mathrm{U}} / A$ and $\left(1-A_{\mathrm{D}}\right) /(1-A)$ in terms of $\gamma$.
% - Hence express $\gamma_{\mathrm{U}}$ in terms of $\{\gamma, V(t), M(t), \xi\}$ and $\gamma_{\mathrm{D}}$ in terms of $\{\gamma, V(t), N(t), S(t), \xi\}$


This section investigate the efficient frontier of the portfolio by setting the greedy value $\gamma$ according to the market.

Let $u$, $u_1$ and $u_2$ be the expected returns of the portfolio, $M(t)$ and $S(t)$. Before the transaction on day $t$, the portfolio's return amounts to $u = Au_1 + (1-A)u_2$, where $A=M(t)/V(t)$ is the proportion of money.

\begin{enumerate}
    \item If the signal of buying is observed, an investor spends $\tilde{m}=\gamma_U M(t)$ buying $(1-\xi)\tilde{m}/S(t)$ shares and changes $u$ to $u_U=A_U u_1 + (1-A_U)u_2$. We have
\begin{align}
    M_U(t)&=M(t)-\tilde{m}=M-\gamma_U M(t)\\
    N_U(t)&=N(t)+\frac{\tilde{m}(1-\xi)}{S(t)}=N(t)+\frac{\gamma_U M(t)(1-\xi)}{S(t)}\\
    V_U(t)&=M(t)-\tilde{m}+\left( N(t)+\frac{\tilde{m}(1-\xi)}{S(t)} \right) S(t)=V(t)-\xi\gamma_U M(t),
\end{align}
and
\begin{align}
    u&=\frac{V_U(t)}{V(t)}=1-\frac{\xi\gamma_U M(t)}{V(t)}\\
    u_1&=\frac{M_U(t)}{M(t)}=1-\gamma_U\\
    u_2&=\frac{N_U(t)}{N(t)}=1+\frac{\gamma_U(1-\xi)M(t)}{N(t)S(t)}.
\end{align}

Then, we can express $A_U$ in terms of $\left\{\gamma_U, V(t), M(t), \xi\right\}$ as
\begin{equation}\label{eq:au}
    A_U=\frac{M_U(t)}{V_U(t)}=\frac{M(t)\left( 1-\gamma_U \right) }{V(t)-\xi\gamma_U M(t)}.
\end{equation}

Similarly, if the signal of selling is observed, an investor sells $\tilde{n}=\gamma_D N(t)$ shares to earn $(1-\xi)\tilde{n}S(t)$ and changes $u$ to $u_D=A_D u_1 + (1-A_D)u_2$. We have
\begin{align}
    M_D(t)&=M(t)+\tilde{n}(1-\xi)S(t)=M(t)+\gamma_D(1-\xi) N(t)S(t)\\
    N_D(t)&=N(t)-\tilde{n}=N(t)-\gamma_D N(t)\\
    V_D(t)&=M_D(t)+N_D(t)S(t)=V(t)-\gamma_D\xi N(t)S(t),
\end{align}
and
\begin{align}
    u&=\frac{V_D(t)}{V(t)}=1-\frac{\gamma_D\xi N(t)S(t)}{V(t)}\\
    u_1&=\frac{M_D(t)}{M(t)}=1+\frac{\gamma_D(1-\xi) N(t)S(t)}{M(t)}\\
    u_2&=\frac{N_D(t)}{N(t)}=1-\gamma_D.
\end{align}
    
Then, we can express $1-A_D$ in terms of $\left\{\gamma_D, V(t), N(t), S(t), \xi\right\}$ as
\begin{equation}\label{eq:ad}
    1-A_D=\frac{N_D(t)S(t)}{V_D(t)}=\frac{N(t)S(t)\left( 1-\gamma_D \right) }{V(t)-\xi\gamma_D N(t) S(t)}.
\end{equation}

\item Then, model $u_U=\gamma(u_2-u)+u$, where $u = Au_1 + (1-A)u_2$ and $u_U = A_U u_1 + (1-A_U)u_2$. We have
\begin{align}
    \gamma&=\frac{A_U u_1+(1-A_U)u_2-u}{u_2-u}\\
    &=1+\frac{A_U(u_1-u_2)}{u_2-u}\\
    \Rightarrow& \frac{A_U}{A}=1-\gamma.\label{eq:au/a}
\end{align}

Similarly, model $u_D=\gamma(u_1-u)+u$, where $u = Au_1 + (1-A)u_2$ and $u_D = A_D u_1 + (1-A_D)u_2$. We have
\begin{align}
    \gamma&=\frac{A_D u_1+(1-A_D)u_2-u}{u_1-u}\\
    &=1+\frac{(1-A_D)(u_2-u_1)}{u_1-Au_1 - (1-A)u_2}\\
    &=1+\frac{(1-A_D)(u_2-u_1)}{(u_1-u_2)(1-A)}\\
    \Rightarrow& \frac{1-A_D}{1-A}=1-\gamma.\label{eq:ad/a}
\end{align}

\item Using Formula \ref{eq:au} and Formula \ref{eq:au/a}, we have
\begin{align}
    \frac{A_U}{A}&=\frac{M(t)(1-\gamma_U)}{\left( V(t)-\xi\gamma_U M(t) \right)} \frac{V(t)}{M(t)}=1-\gamma\\
    \Rightarrow & \gamma_U=\frac{\gamma V(t)}{V(t)+(\gamma-1)\xi M(t)}.
\end{align}

Similarly, using Formula \ref{eq:ad} and Formula \ref{eq:ad/a}, we have
\begin{align}
    \frac{1-A_D}{1-A}&=\frac{N(t)S(t)(1-\gamma_D)}{\left( V(t)-\xi\gamma_D N(t)S(t) \right)} \frac{V(t)}{N(t)S(t)}=1-\gamma\\
    \Rightarrow & \gamma_D=\frac{\gamma V(t)}{V(t)+(\gamma-1)\xi N(t)S(t)}.
\end{align}

\item 
\begin{figure}[!htbp]
    \begin{center}
        \includegraphics[width=0.8\textwidth]{../code/img/q10.pdf}
    \end{center}
    \caption{}
    \label{fig:q10}
\end{figure}
\end{enumerate}

\section{Adaptive Greed}\label{sec:11}
$\gamma_i^*=\underset{\gamma}{\operatorname{argmax}}\left[\frac{V\left(t_i ; \gamma\right)}{V\left(t_i-1 ; \gamma\right)}\right]$ be the optimal $\gamma$ at time $t_i$. 



\begin{figure}[!htbp]
    \begin{center}
        \includegraphics[width=0.8\textwidth]{../code/img/q11_1.pdf}
    \end{center}
    \caption{Visualization of portfolio's value with different $\gamma$ and $\gamma^*$. The black point is the selected $\gamma^*$ at time $t$.}
    \label{fig:q11-1}
\end{figure}


\begin{equation}
    R[t]=\frac{V_A[t]-V_A[t-k]}{V_C[t]-V_C[t-k]}\frac{V_C[t]}{V_A[t]}.
\end{equation}

Then, Bob's $\gamma_B$ of time $t$ is
\begin{equation}
    \gamma_B[t] =
    \left\{\begin{array}{ll}
        \gamma_A, & \text{if}\quad R[t]>1 \\
        \gamma_C & else.
    \end{array}\right.
\end{equation}



\begin{figure}[!htbp]
    \begin{center}
        \includegraphics[width=0.8\textwidth]{../code/img/q11_2.pdf}
    \end{center}
    \caption{The portfolio's value of Alice, Bob and Charlie as well as the chosen $\gamma_B$. The red point is the selected $\gamma_B$ at time $t$.}
    \label{fig:q11-2}
\end{figure}

\newpage
\bibliographystyle{splncs04}
\bibliography{refs}

\section*{Acknowledgements} I really appeciate JIANG Wenxin for finishing the work from question 1-5 and 11.2. 
She is highly motivated, so she couldn't wait me to work on question 11.2 and finished code of q11.2 on her own. \\
Meanwhile, I also have to thank Github-Copilot for assisting me to write this latex report.
% \section*{Appendix}
% \lstinputlisting[caption=Codes of the Analysis, label={codes}, language=Python]{./codes.py}
\end{document}